% !TeX spellcheck = de_DE
\documentclass[twocolumn]{article}

\usepackage[german]{babel}
\usepackage{microtype}
\usepackage{verse}
\usepackage{parskip}
\usepackage{hyperref}

\newcommand{\titlevar}{Ein Liebesgedicht}
\newcommand{\authorvar}{Günthner}
\newcommand{\datevar}{Winter 2024}
\title{\vspace{-3cm}\titlevar}
\author{\authorvar}
\date{\datevar}
\hypersetup{
	pdftitle=\titlevar,
	pdfauthor=\authorvar,
	pdfcreationdate=\datevar,
}
\setlength{\parindent}{0pt}
\newcommand{\neueStrophe}{\rule{0pt}{0pt}\\}

\begin{document}
	\maketitle
	
	\begin{verse}
		Was soll ich sagen \\
		Es bedeutet mir viel \\
		Dass wir uns haben \\
	\end{verse}
	
	\begin{verse}	
		Es gibt kein Wort \\
		Das es beschreibt \\
		Dieses Gefühl \\
		Das immer bleibt \\
	\end{verse}
	
	\begin{verse}	
		Jetzt weiß ich's \\
		Pass auf! \\
	\end{verse}
	
	\begin{verse}	
		Zwei Seelen \\
		Brennen \\
		Brennen allein \\
		Fürchten sich sehr \\
		Finden zusammen \\
		Fürchten nicht mehr \\
	\end{verse}
	
	\begin{verse}	
		Vier Sonnen \\
		Sich gegenüber \\
		Die sich kennen \\
		Schon immer \\
	\end{verse}
	
	\begin{verse}	
		Kannst du ihn hören \\
		Ihren Gesang \\
		Sie singen alleine \\
		Sie singen zusamm' \\
	\end{verse}
	
	\begin{verse}	
		Aber warum allein? \\
		Sie sind doch zu zweit? \\
		Die Antwort darauf \\
		Liegt in ihrer Zweisamkeit \\
	\end{verse}
	
	\begin{verse}	
		Sie sind am verschmelzen \\
		Diese \textbf{hier} Menschen \\
		Wie gegossen \\
		In dieselbe Form \\
	\end{verse}
	
	\begin{verse}	
		Ich liebe dich, Julia \\
		Für alle Gezeiten \\
		Ich erwarte mit Freuden \\
		Unser Jubeljahr \\
	\end{verse}
\end{document}